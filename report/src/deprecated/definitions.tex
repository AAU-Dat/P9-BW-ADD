\section{Definitions}\label{sec:definitions}
\begin{definition}[Markov Chain]
    A Markov chain is a tuple $\mathcal{M} = (S, \mathcal{L}, \mathscr{l}, \tau, \pi)$, where:
    \begin{itemize}
        \item $S$ is a finite set of states.
        \item $\mathcal{L}$ is a finite set of labels.
        \item $\mathscr{l}: S \rightarrow \mathcal{L}$ is a labeling function, which assigns a label to each state.
        \item $\tau: S \rightarrow \mathcal{D}(S)$ is a transition function.
        The model moves from state $s$ to state $s'$ with probability $\tau(s, s')$.
        \item $\pi$: is the initial distribution, the model starts in state $s$ with probability $\pi(s)$.
    \end{itemize}
\end{definition}


Intuitively, a Markov chain is a model that starts in a state $s$ with probability $\pi(s)$, and then transitions to a new state $s'$ with probability $\tau(s, s')$. 
The model continues to transition between states according to the transition function.


\begin{definition}[Hidden Markov Model]
    A Hidden Markov Model (HMM) is a tuple $\mathcal{M} = (S, \mathcal{L}, \mathscr{l}, \tau,  \pi)$, where $S, \mathcal{L}, \tau, \pi$
    are defined as above, and:
    \begin{itemize}
        \item $\mathscr{l}: S \rightarrow D(\mathcal{L})$ is the emission function.
        The model emits a label $l$ in state $s$ with probability $\mathscr{l}(s, l)$.
    \end{itemize}
\end{definition}


Intuitively, an HMM is a model that starts in a state $s$ with probability $\pi(s)$, then emits a label $l$ with probability $\mathscr{l}(s, l)$, and transitions to a new state $s'$ with probability $\tau(s, s')$.
The model continues to emit labels and transition between states according to the emission and transition functions.


\begin{definition}[Markov Decision Process]
    A Markov Decision Process (MDP) is a tuple $\mathcal{M} = (S, \mathcal{L}, \mathscr{l}, A, \{\tau_a\}_{a \in A}, \pi)$ where $S, \mathcal{L}, \mathscr{l}, \pi$ are defined as above, and:
    \begin{itemize}
        \item $A$ is a finite nonempty set of actions.
        \item $\tau_a: S \rightarrow \mathcal{D}(S)$ is a transition function for each action $a \in A$. The model moves from state $s$ to state $s'$ with probability $\tau_a(s, s')$ when action $a$ is taken.
    \end{itemize}
\end{definition}


Intuitively, an MDP is a model that starts in a state $s$ with probability $\pi(s)$, then emits a label $\mathscr{l}(s)$ and, it can recieve an action $a \in A$ and transition to a new state $s'$ with probability $\tau_a(s, s')$.


\subsection{Continuous-Time}
In the previous definitions, the models are discrete-time models, where time advances in fixed, regular steps. 
For example, in a discrete-time Markov chain, the system transitions between states at each step or tick of a clock, and the probability of moving from one state to another is governed by the transition function $\tau(s, s')$. 
This means that transitions can only happen at specific time intervals (e.g., after every second, every minute, etc.).

In contrast, continuous-time models allow transitions to occur at any time, rather than at fixed intervals. 
The time between transitions is variable and follows a continuous distribution. 
This introduces the concept of transition rates rather than discrete transition probabilities.


\begin{definition}[Continuous-Time Markov Chain]
    A Continuous-Time Markov Chain (CTMC) is a tuple $\mathcal{M} = (S, \mathcal{L}, \mathscr{l}, R, \pi)$, where $S, \mathcal{L}, \mathscr{l}, \pi$ are defined as above, and:
    \begin{itemize}
        \item $R: S \times S \rightarrow \mathbb{R}_{\geq 0}$ is the rate function. The model transitions from state $s$ to state $s'$ with rate $R(s, s')$.
    \end{itemize}
\end{definition}


For two states $s$ and $s'$, $R(s, s')$ gives the rate at which the system moves from state $s$ to state $s'$. 
A higher rate means a faster transition.

A Continuous-Time Markov Chain (CTMC) is a type of Markov model where the time between transitions is not fixed but is governed by exponential distributions. 
If there are more then one outgoing transition from a state, we get race-conditions, the first transition to occur is the one that will be taken. 
The time spent in a state before transitioning to a new state is called $dwell-time$. 
This is exponentially distributed with a rate $E(s) = \sum_{s' \in S} R(s, s')$. 
The probability of transitioning from state $s$ to state $s'$ is $R(s, s')/E(s)$, the time spent in $s$ is independent from the properbility of transitioning to $s'$.

\subsection{Matrix Representation}
The transition function $\tau$ can be represented as a matrix, where each element $\tau(s, s')$ is the probability of transitioning from state $s$ to state $s'$. 
The matrix representation of $\tau$ is called the transition matrix. 
The transition matrix is a square matrix with dimensions $|S| \times |S|$, where $|S|$ is the number of states in the model. 
The transition matrix is a stochastic matrix, meaning that the sum of each row is equal to 1, meaning all the probabilities of transitioning from state $s$ to all other states sum to 1.

If we take an example of a model with two states $S = \{s_1, s_2\}$, the transition matrix $\tau$ is defined as:
\begin{equation}
    \tau = \begin{bmatrix}
        \tau(s_1, s_1) & \tau(s_1, s_2) \\
        \tau(s_2, s_1) & \tau(s_2, s_2)
    \end{bmatrix}
\end{equation}


We can give an example of a transition matrix for a model with two states, where the model transitions from state $s_1$ to state $s_2$ with probability 0.4 and transitions from state $s_2$ to state $s_1$ with probability 0.5:
\begin{equation}
    \tau = \begin{bmatrix}
        0.6 & 0.4 \\
        0.5 & 0.5
    \end{bmatrix}
\end{equation}

The initial distribution $\pi$ is a vector that represents the probability of starting in each state. 
The initial distribution is a stochastic vector, meaning that the sum of all probabilities is equal to 1. 
The initial distribution $\pi$ is a vector with dimensions $|S|$, where $|S|$ is the number of states in the model. 
Each element $\pi(s)$ is the probability of starting in state $s$.
\begin{equation}
    \pi = \begin{bmatrix}
        0.6 \\
        0.5
    \end{bmatrix}
\end{equation}

The labeling function $\mathscr{l}$ can be represented as a matrix, where each element $\mathscr{l}(s, l)$ is the probability of emitting label $l$ in state $s$.
The matrix representation of $\mathscr{l}$ is called the emission matrix.
The emission matrix is a matrix with dimensions $|S| \times |\mathcal{L}|$, where $|\mathcal{L}|$ is the number of labels in the model.
The emission matrix is a stochastic matrix, meaning that the sum of each row is equal to 1, meaning all the probabilities of emitting a label in state $s$ sum to 1. 

If we take an example of a model with two states $S = \{s_1, s_2\}$ and two labels $\mathcal{L} = \{l_1, l_2\}$, the emission matrix $\mathscr{l}$ is defined as:
\begin{equation}
    \mathscr{l} = \begin{bmatrix}
        \mathscr{l}(s_1, l_1) & \mathscr{l}(s_1, l_2) \\
        \mathscr{l}(s_2, l_1) & \mathscr{l}(s_2, l_2)
    \end{bmatrix}
\end{equation}
We can give an example of an emission matrix for a model with two states and two labels, where the model emits label $l_1$ in state $s_1$ with probability 0.7 and emits label $l_2$ in state $s_2$ with probability 0.6:
\begin{equation}
    \mathscr{l} = \begin{bmatrix}
        0.7 & 0.3 \\
        0.4 & 0.6
    \end{bmatrix}
\end{equation}

The rate function $R$ can be represented as a matrix, where each element $R(s, s')$ is the rate of transitioning from state $s$ to state $s'$. 
The matrix representation of $R$ is called the rate matrix. 
The rate matrix is a square matrix with dimensions $|S| \times |S|$, where $|S|$ is the number of states in the model. 
The rate matrix is a non-negative matrix, meaning that all elements are greater than or equal to 0.
\begin{equation}
    R = \begin{bmatrix}
        R(s_1, s_1) & R(s_1, s_2) \\
        R(s_2, s_1) & R(s_2, s_2)
    \end{bmatrix}
\end{equation}

If we take an example of a model with two states $S = \{s_1, s_2\}$, the rate matrix $R$ is defined as:
\begin{equation}
    R = \begin{bmatrix}
        0.5 & 0.3 \\
        0.2 & 0.4
    \end{bmatrix}   
\end{equation}