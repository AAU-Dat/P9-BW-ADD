\section{Future Works}\label{sec:future_works}
This section outlines potential future works and extensions of the symbolic Baum-Welch algorithm.

Since no experiments were conducted on the implementation, this is something that would be a good starting point for future work, as we could compare the performance of the symbolic approach to the recursive Jajapy implementation.
This comparison would assess the scalability of the symbolic implementation as the number of model states increases. 
Runtime and memory usage would be core parameters to evaluate, providing insights into the computational efficiency of the symbolic approach in the context of the Baum-Welch algorithm.
Additionally, we intend to test the symbolic approach on concrete models of varying sizes and complexities. 

To deepen our analysis, we would compare the symbolic implementation with a C++-based recursive implementation of the Baum-Welch algorithm. 
In \cite{p7} the authors implemented the forward-backward algorithm in C and a symbolic approach in C, which showed that the symbolic approach was only a little faster than the C implementation.
Therefore it would be interesting to compare the full symbolic approach to a recursive approach in C++.
By benchmarking both implementations within the same programming language, we could better understand the symbolic approach's performance.

A key aspect in our project was integrating the symbolic Baum-Welch algorithm into the Storm model checker. 
Leveraging the Storm Parser Library, we planed to enable the parsing of models and observation sequences for the Baum-Welch algorithm.
This integration would expand the algorithm's accessibility and applicability, as Storm is widely used in model-checking research and practice.

We would also explore extending the symbolic approach to \glspl{mdp}. 
Given the close relationship between \glspl{mdp} and \glspl{hmm}, this extension would allow the symbolic Baum-Welch algorithm to estimate \gls{mdp} parameters. 
Such advancements would broaden its utility in diverse applications.

Another area of improvement would implementing the Baum-Welch algorithm using sparse matrices. 
We would evaluate their respective runtime and memory efficiency by comparing this traditional approach to the symbolic implementation. 

Finally, we would investigate alternative symbolic representations beyond \glspl{add}. 
Options like Multi-valued Decision Diagrams (MDDs), which support multiple edges per node.
Comparative experiments will assess these representations runtime and memory usage against \glspl{add}, offering further insights into their relative strengths and weaknesses.