\section{Introduction}\label{sec:introduction}
\glspl{hmm} are probabilistic models widely used to model systems with hidden states. 
Applications span speech recognition, bioinformatics and finance~\cite{chavan2013overview,ciocchetta2009bio, mamon2007hidden}, where systems exhibit stochastic behavior. 
\glspl{hmm} rely on the Markov property, which assumes that the system's future depends only on its current state~\cite{baum1966statistical}. 
This memory-less behavior simplifies analysis and makes \glspl{hmm} particularly effective in modeling systems where the underlying states are not directly observable.

Parameter estimation is a critical task in analyzing \glspl{hmm}, as the accuracy of model predictions depends heavily on the quality of the estimated parameters~\cite{bacci2023mm}. 
The Baum-Welch algorithm, a widely used Expectation-Maximization (EM) framework~\cite{kenny2014deep}, iteratively refines \gls{hmm} parameters to maximize the likelihood of observed data~\cite{levinson1983introduction}. 
However, traditional algorithm implementations rely on matrix representations of the model, which suffer from quadratic space complexity, limiting scalability for large systems~\cite{davis2004comparing}.

In probabilistic verification, tools such as PRISM~\cite{kwiatkowska2011prism} and Storm~\cite{hensel2021probabilistic} provide state-of-the-art model-checking capabilities for Markov models. 
These tools efficiently verify properties such as reachability and safety using symbolic representations, such as \glspl{add}, to handle large state spaces. 
Despite their advanced capabilities, these tools lack native support for parameter estimation, which is crucial for learning models from data. 
Conversely, tools like Jajapy~\cite{reynouard2023jajapy} and SUDD~\cite{p7} focus on parameter estimation but lack integration with symbolic model-checking frameworks.

By leveraging \glspl{add} in all algorithm steps for the Baum-Welch algorithm, we achieve significant improvements in runtime performance and scalability compared to matrix-based approaches. 
Our work builds on the symbolic techniques introduced in SUDD but expands their application to the entire parameter estimation process for \glspl{hmm}. 

Our contributions are as follows:
\begin{itemize}
    \item We implement the complete Baum-Welch algorithm symbolically, including forward-backward computations and parameter updates.
    \item We provide a comparative evaluation of our approach against Jajapy.
\end{itemize}

\subsection{Related Works}\label{subsec:related-works}
Several tools have been developed for parameter estimation in Markov models. 
Jajapy is a Python-based tool that uses the Baum-Welch algorithm to estimate parameters in various models, including \glspl{hmm}~\cite{reynouard2023jajapy}. 
It employs a matrix representation of the model and implements the necessary operations for parameter estimation through standard matrix computations without standard matrix libraries, such as NumPy~\cite{harris2020array}.

While accessible and straightforward, Jajapy is hindered by the space complexity inherent in its iterative-based calculation.
This limitation makes it computationally expensive for large-scale models, as memory requirements grow quadratically with the number of states in the system.
While versatile, Jajapy relies on matrix representations, resulting in quadratic space complexity~\cite{davis2004comparing}.

SUDD builds upon the limitations of Jajapy by introducing a symbolic representation for the forward-backward algorithm for calculating \glspl{pctmc}, it does not handle the update step in the Baum-Welch algorithm~\cite{p7}.
Specifically, it leverages \glspl{add} to reduce memory consumption and improve the runtime performance of the Baum-Welch algorithm.
By employing \gls{add}-based computations, SUDD significantly improves scalability, making it feasible to handle larger models.

However, the update step in the Baum-Welch algorithm requires SUDD to give an explicit state space representation of the model, which limits the algorithm's scalability.

Our work extends the capabilities of SUDD by implementing the complete Baum-Welch algorithm for \glspl{hmm} using symbolic representations.
Here by we provide a method that estimates parameters efficiently and enables the accurate modeling of complex systems.

%In this paper, we extend the work of SUDD by utilizing \glspl{add} in all the steps of the Baum-Welch algorithm to improve the runtime performance of parameter estimation for \gls{hmm}.
%Our approach is to extend Storm with parameter estimation capabilities, by integrating the Baum-Welch algorithm with symbolic representations to improve the runtime performance of parameter estimation for \glspl{hmm}.


% Our approach not only inherits the scalability benefits of \glspl{add} but also implements the complete parameter estimation process.
% Additionally, we compare our implementation with both the original Jajapy, SUDD and an extended version of SUDD using the log-semiring framework, which improves numerical stability in computations.

%By integrating scalable symbolic representations into the full Baum-Welch algorithm, we provide a method that not only estimates parameters efficiently but also enables the accurate modeling of complex systems.


%Model checking
%parameter estimation
%markovian models
%what are they used for (applications)
%what are the challenges

%related works
%what are the current solutions
%P7
%Jajapy
%Prism 
%Storm


% Use the definition LaTeX environment, and base the definitions off of Raphaels paper.


%Describe the field we are in
%Markov models are characterized by the Markov property, which states that the future behavior of the system depends only on its current state and not on its past history. 
%This property makes Markov models well-suited for modeling systems with memoryless behavior, where the future state of the system is independent of its past states given the current state.

%The need for correct parameter estimation in Markov models is crucial for making accurate predictions about the system's behavior and for verifying the validity of the model.
%Probabilistic model checking tools such as PRISM~\cite{kwiatkowska2011prism} and Storm~\cite{dehnert2017storm} are widely used in the verification of Markov models.
%These tools are used to analyze the behavior of the system and to check if the model satisfies certain properties, such as reachability and safety properties.

% For finding the correct properties of the model, the parameters of the model need to be estimated from data, a process known as parameter estimation.
% Parameter estimation is a crucial step in the analysis of Markov models, as it allows us to learn the underlying dynamics of the system and make predictions about its future behavior.
% The Baum-Welch algorithm is a popular method for estimating the parameters for Markov Models~\cite{baum1970maximization}.
% The algorithm uses the Expectation-Maximization (EM) framework to iteratively update the parameters of the model until convergence.
% Normal implementations of the Baum-Welch algorithm uses matrices to represent the model, which has a space complexity that grows quadratically with the number of states in the model. 
% This makes the algorithm computationally expensive for large models, as the memory requirements grow rapidly with the size of the model. 
% The model checkers meantioned before, Storm and Prism, use symbolic data structures such as Algebraic Decision Diagrams (ADD)s to represent the model, which allows them to handle large models efficiently. 
% However, the Baum-Welch algorithm does not take advantage of these symbolic data structures, which limits its scalability for large models.
% This paper aims to improve the runtime of the Baum-Welch algorithm by using a symbolic representation of the model, and use symbolic operations to perform the operations required by the algorithm.

%This paper is about improving the runtime of Jajapy - a tool for estimating parameters in parametric models~\cite{goossens1993}.
%Introduce Markov models
% Markov models are a class of probabilistic models that are used to describe the evolution of a system over time.
% A Markov model has the Markov property, which states that the future behavior of the system depends only on its current state and not on its past history~\cite{markov1962theory}.
% This property simplifies analysis by focusing only on the present state, making Markov models especially useful for systems where memory-less behavior is a reasonable assumption.

% Markov models are widely used in various fields, such as biology, finance, and computer science, to model systems that exhibit stochastic behavior~\cite{covid19_prism,ciocchetta2009bio, mamon2007hidden,lazowska1984quantitative}.
% As such, their analysis has a wide range of applications.

% An example of a Markov model, is a simple weather model, if today is sunny, there might be an 80\% chance of sun tomorrow and a 20\% chance of rain.
% Similarly, if today is rainy, there might be a 70\% chance of rain tomorrow and a 30\% chance of sun.

% %Model checking
% Model checking is a technique used to verify the correctness of Markov models by comparing the predictions of the model with observed data.
% Model checking is widely used in the verification of Markov models, where the model is analyzed to check if it satisfies certain properties~\cite{clarke1997model}.
% It ensures reliability and correctness in critical systems, from traffic controls to industrial automation and communication protocols~\cite{clarke1997model}.
% It is also used to check if the model satisfies certain properties, such as reachability, can we reach a desired state and safety properties, can we avoid going a specific sequence of states.

% A real world example of model checking is the verification of a traffic light system, where the model is analyzed to check if the traffic lights are working correctly.
% For reachability, we can ask: \textit{can a traffic light system always cycle back to green after being red?}.
% For safety properties we can ask, \textit{can a traffic light system avoid having both lights green at the same time?}.

% There exists several tools for model checking, such as PRISM~\cite{kwiatkowska2011prism} and Storm~\cite{hensel2021probabilistic}, which are widely used in the verification of Markov models.
% These tools use symbolic representations to represent the model and perform the operations required for model checking.
% The limitation of these tools is that they do not support parameter estimation, which makes them unsuitable for learning the parameters of the model from data.

% Parameter estimation is a crucial step in the analysis of Markov models, as the analysis of the model depends on the accuracy of the estimated parameters, particularly when in a timing and probabilistic behaviour~\cite{bacci2023mm}.
% Parameter estimation is the process of estimating the parameters of the model from observed data, which is used to make predictions about the system's behavior.

% These parameters are used to ensure that the model accurately represents the system's behavior and dynamics and to make accurate predictions about the system's future behavior.
% Accurate parameter estimation is essential for making reliable predictions and validating model behavior, with applications ranging from healthcare diagnostics to network security~\cite{bacci2023mm}.

% The Baum-Welch algorithm is a widely used method for estimating the parameters of Markov models~\cite{kenny2014deep}.
% The algorithm uses the Expectation-Maximization (EM) framework to iteratively update the parameters of the model until convergence~\cite{levinson1983introduction}.
% The Baum-Welch algorithm is computationally expensive for large models, as it uses matrices to represent the model, which has a space complexity that grows quadratically with the number of states in the model.
% This makes the algorithm computationally expensive for large models, as the memory requirements grow rapidly with the size of the model~\cite{davis2004comparing}.

% Addressing these challenges requires innovative techniques, such as symbolic representations, which reduce memory consumption while preserving accuracy.

% %PRISM
% PRISM has developed a language for specifying models and properties, called the PRISM Language, which is widely used in the field of probabilistic model checking.

% When models are specified in the PRISM Language, PRISM can provide a symbolic representation such as \glspl{add} to represent and manipulate the models efficiently, enabling the verification of properties like reachability and safety.

% However, PRISM does not support parameter estimation, making it unsuitable for tasks requiring the inference of model parameters from observed data.

% %Storm


% It has a parser to read models specified in the PRISM Language, making it easy to use for users familiar with PRISM.
% Storm has been optimized for scalability and flexibility, supporting a wide range of model types and verification tasks.

% Additionally, Storm is open-source and has a large user base, making it a popular choice for probabilistic model checking.
% Despite these strengths, Storm also lacks support for parameter estimation, limiting its utility for learning model parameters from data.

