\section{Preliminaries}\label{sec:preliminaries}
In this section, we introduce the necessary background concepts and definitions that are essential for understanding the subsequent sections.
We begin by defining the key concepts of a \gls{hmm} and then describe how a \gls{hmm} can be represented using matrices.
We then introduce the Baum-Welch algorithm, which is used to estimate the parameters of a \gls{hmm} from observed data.
We describe all the steps involved in the Baum-Welch algorithm, including the forward-backward algorithm and the update algorithm.
Finally, we discuss how the Baum-Welch algorithm can be implemented using matrix operations to efficiently compute the forward and backward variables, intermediate variables, and parameter updates.

\subsection{Hidden Markov Models}\label{subsec:hmm}
%definition of HMM
\glspl{hmm} were introduced by Baum and Petrie in 1966~\cite{baum1966statistical}.
\gls{hmm} are a class of probabilistic graphical models that are widely used to model sequences of observations with underlying hidden states.
These models consist of two main components: observations and hidden states.
The observations are the visible data emitted by the model, while the hidden states represent the underlying process that generates these observations.
The objective of an \gls{hmm} is to infer the hidden states based on the observations.
\glspl{hmm} have applications in fields such as speech recognition~\cite{chavan2013overview}, bioinformatics~\cite{de2007hidden}, and natural language processing~\cite{murveit1990integrating}.
\glspl{hmm} was chosen as the model of choice for this project due to its versatility and ability to model complex systems.

\begin{definition}[Hidden Markov Model]
    A Hidden Markov Model (HMM) is a tuple $\mathcal{M} = (S, \mathcal{L}, \mathscr{l}, \tau,  \pi)$, where:
    \begin{itemize}
        \item $S$ is a finite set of states.
        \item $\mathcal{L}$ is a finite set of labels.
        \item $\mathscr{l}: S \rightarrow D(\mathcal{L})$ is the emission function.
        \item $\tau: S \rightarrow D(S)$ is the transition function.
        \item $\pi: S \rightarrow \mathbb{R}$ is the initial distribution.
    \end{itemize}
\end{definition}

Here, $D()$ denotes the set of probability distributions over a finite set. The model emits a label $l$ in state $s$ with probability $\mathscr{l}(s, l)$, transitions between states with probability $\tau(s, s')$, and starts in state $s$ with probability $\pi(s)$.

%should i write about it is descrite time or not?
\glspl{hmm} can be further classified into discrete-time and continuous-time models based on the time scale of the underlying process.
Discrete-time \glspl{hmm} are the most common type of \glspl{hmm} and are used to model sequences of observations that occur at discrete time intervals.
Continuous-time \glspl{hmm} are used to model sequences of observations that occur continuously over time.
In this project, we focus on discrete-time \glspl{hmm}.

%HMM as matrices
\subsection{Matrix Representation of HMMs}\label{subsec:matrix-representation}
\glspl{hmm} can be represented using matrices.
The emission function $\mathscr{l}$ can be represented as a matrix $\omega$ where $\omega_{s, l} = \mathscr{l}(s, l)$.
The matrix $\omega$ has the size $|S| \times |\mathcal{L}|$.
The sum of each row in the matrix $\omega$ is equal to one, reflecting the total probability of emitting all labels from a given state.
\[
\omega = \begin{bmatrix}
    \mathscr{l}(s_1, l_1) & \cdots & \mathscr{l}(s_1, l_{|\mathcal{L}|}) \\
    \vdots & \ddots & \vdots \\
    \mathscr{l}(s_{|S|}, l_1)  & \cdots & \mathscr{l}(s_{|S|}, l_{|\mathcal{L}|})
\end{bmatrix}
\]
The transition function $\tau$ can be represented as a matrix $P$ where $P_{s, s'} = \tau(s, s')$.
The matrix $P$ has the size $|S| \times |S|$.
The sum of each row in $P$ is equal to one, reflecting the total probability of transitioning from a given state to all other states.
\[
P = \begin{bmatrix}
    \tau(s_1, s_1) &  \cdots & \tau(s_1, s_{|S|}) \\
    \vdots  & \ddots & \vdots \\
    \tau(s_{|S|}, s_1) & \cdots & \tau(s_{|S|}, s_{|S|})
\end{bmatrix}
\]
The initial distribution $\pi$ can be represented as a vector $\pi$ where $\pi_s = \pi(s)$.
The vector $\pi$ has the size $|S|$.
The sum of all elements in $\pi$ is equal to one, reflecting the total probability of starting in each state.
\[
\pi = \begin{bmatrix}
    \pi(s_1) \\
    \vdots \\
    \pi(s_{|S|})
\end{bmatrix} 
\]

\subsection{Observations and Hidden States}\label{subsec:observations-hidden-states}
An \gls{hmm} operates on sequences of observations, denoted as $O = {O_1, O_2, \ldots, O_N}$, where each $O_i$ is a sequence of labels $o_1, o_2, \ldots, o_{|\textbf{O}|-1}$.
The task is to infer the sequence of hidden states $S = s_1, s_2, \ldots, s_{|\textbf{O}|-1}$ that most likely generated these observations.

Given an observation sequence $O$, the goal is to maximize the probability of the hidden states conditioned on the observations.

This inference is commonly achieved using the Baum-Welch algorithm.

%Baum-Welch for HMM mathamatically
\subsection{Baum-Welch Algorithm}\label{subsec:baum-welch}
The Baum-Welch algorithm is a key method for estimating the parameters of an \gls{hmm} from observed data. 
It was chosen as the method of choice for this project due to its ability to estimate the parameters of a \gls{hmm} without knowing the hidden states that generated the observations, and it is also the standard method for training \glspl{hmm}.
If looking at other Markov models such as \glspl{mc}, the Baum-Welch algorithm can be used to estimate the parameters of the model from observed data, therefore it is a suitable choice for this project, as this can be used to estimate the parameters of other Markov models.
It leverages the Expectation-Maximization (EM) framework and consists of two iterative steps:

\begin{enumerate}
    \item \textbf{Expectation Step (E-step)}: Compute the expected the forward and backward variables, for each state $s$ and time $t$. of the latent variables, which are the unobserved state sequences corresponding to the observations. These variables represent the likelihood of being in state $s$ at time $t$ given the observed data up to time $t$ and the likelihood of observing the remaining data from time $t$ onwards given the state $s$ at time $t$, respectivly. 
    \item \textbf{Maximization Step (M-step)}: Update the model parameters (emission matrix $\omega$, transition matrix $P$, and initial distribution $\pi$) to maximize the likelihood of the observed data, using the expected values computed in the E-step.
    \item Repeat the E-step and M-step until convergence.
\end{enumerate}

The Baum-Welch algorithm is particularly useful for estimating the properbilities of the emission and transition matrices of a HMM, given a set of observations, without knowing the hidden states that generated the observations.

Given a multiset of observations $\mathcal{O}$ and initial parameters $\textbf{x}_0$, the Baum-Welch algorithm estimates the parameters of a \gls{hmm} $\mathcal{P}$ by iteratively improving the current hypothesis $\textbf{x}_n$ using the previous estimate $\textbf{x}_{n-1}$ until a convergence criterion is met.
A hypothesis refers to a specific set of values for the parameters $\mathbf{x}$.

Each iteration of the algorithm produces a new hypothesis, denoted as $\textbf{x}_n$, which is the algorithm's current best guess for the parameter values based on the observed data.
The algorithm consists of three main steps: the forward-backward procedure, the update step, and the convergence criterion.
The Baum-Welch algorithm iteratively refines the parameters until the improvement between successive iterations falls below a predefined threshold.
This is typically evaluated using a convergence criterion such as:

\begin{equation}
    ||l(\textbf{x}_n) - l(\textbf{x}_{n-1})|| < \epsilon\label{eq:convergence-criterion}
\end{equation}

where $\epsilon > 0$ is a small threshold, and $l(\textbf{x}_n)$ denotes the likelihood of the observed data given the parameter values at the $n$-th iteration.

The algorithm stops when the change in parameters is sufficiently small, indicating that the model has converged to a local maximum of the likelihood function.
The parameter estimation procedure is outlined in \autoref{alg:parameter-estimation}.

\begin{algorithm}[htb!]
    \begin{codebox}
        \Procname{$\proc{Estimate-Parameters}(\mathcal{P}, \mathbf{x}_0, \mathcal{O})$}
        \li $\mathbf{x} \gets \mathbf{x}0$
        \li \While $\neg\proc{Criterion}(\mathbf{x}_{n-1}, \mathbf{x}_n)$
        \li \Do $\mathbf{x}_{n - 1} \gets \mathbf{x}_n$
        \li $(\alpha, \beta) = \proc{Forward-Backward}(\mathcal{P}(\mathbf{x}_n), \mathcal{O})$
        \li $\mathbf{x}_n = \proc{Update}(\mathcal{P}(\mathbf{x}_n), \mathcal{O}, \alpha, \beta)$ \End
        \li \Return $\mathbf{x}_n$
    \end{codebox}
    \caption{Parameter estimation procedure~\cite{p7}.}
    \label{alg:parameter-estimation}
\end{algorithm}

Starting with initial parameters $\mathbf{x}_0$, the parameter estimation procedure iteratively improves the current hypothesis $\mathbf{x}_n$ using the previous estimate $\mathbf{x}_{n-1}$ until a specified criterion for convergence is met, the algorithm returns the final estimate $\mathbf{x}_n$.
The specifics of the $\proc{Forward-Backward}$ and $\proc{Update}$ procedures are detailed in \autoref{subsec:forward-backwards_algorithm} and \autoref{subsec:update-algorithm} from~\cite{baum1970maximization}.

\subsection{Forward-Backward Algorithm}\label{subsec:forward-backwards_algorithm}
%Forward-Backward algorithm
For a given \gls{hmm} $\mathcal{M}$, the forward-backward algorithm computes the forward and backward variables, $\alpha_s(t)$ and $\beta_s(t)$, for each observation sequence $o_0, o_1, \dots, o_{|\mathbf{o}|-1} = \mathbf{o} \in \mathcal{O}$.
The forward variable $\alpha_s(t)$ represents the likelihood of observing the partial sequence $o_0, o_1, \dots, o_t$ and being in state $s$ at time $t$, given the model $\mathcal{M}$.
The backward variable $\beta_s(t)$ represents the likelihood of observing the partial sequence $o_{t+1}, o_{t+2}, \dots, o_{|\mathbf{o}|-1}$ given state $s$ at time $t$ and the model $\mathcal{M}$.


% \begin{equation}
%     \alpha_s(t) = l(o_0, o_1, \dots, o_t, S_{t} = s \mid \mathcal{M})
%     \label{eq:alpha-recursive}
% \end{equation}


% \begin{equation}
%     \beta_s(t) = l(o_{t+1}, o_{t+2}, \dots, o_{|\mathbf{o}|-1} \mid S_{t} = s, \mathcal{M})
%     \label{eq:beta-recursive}
% \end{equation}


The forward variable $\alpha_s(t)$ and backward variable $\beta_s(t)$ can be computed recursively as follows:


\begin{equation}
    \alpha_s(t) =
    \begin{cases}
        \omega_s(0) \; \pi_s & \text{if } t = 0 \\
        \omega_s(t) \sum_{s' \in S} P_{s's}\alpha_{s'}(t - 1) & \text{if } 0 < t \leq |\mathbf{o}| - 1 \\
    \end{cases}
    \label{eq:forward-recursive}
\end{equation}


\begin{equation}
    \beta_s(t) =
    \begin{cases}
        \mathbbm{1} & \text{if } t = |\mathbf{o}| - 1 \\
        \sum_{s' \in S} P_{ss'} \omega_{s'}(t + 1) \beta_{s'}(t + 1) & \text{if } 0 \leq t < |\mathbf{o}| - 1 \\
    \end{cases}
    \label{eq:backward-recursive}
\end{equation}


Here, $\omega_{s}(t)$ is the likelihood of observing $o_t$ given that the state at time $t$ is $s$ and the model $\mathcal{M}$, formally $\omega_s(t) = l(o_t \mid S_t = s, \mathcal{M})$.
Meaning that $\omega_s(t)$ is the probability of observing the label $o_t$ in state $s$.

The forward-backward algorithm computes the forward and backward variables for each state $s$ and time $t$ in the observation sequence $\mathbf{o}$, providing a comprehensive view of the likelihood of the observed data under the model.

In preparation for later discussions we would like to draw the attention to the fact that the above recurrences can be solved using dynamic programming requiring one to use $\Theta(|S|\times|(|\mathbf{o}|-1)|)$ space.


\subsection{Update Algorithm}\label{subsec:update-algorithm}
%Update of HMM
The update algorithm refines the parameter values of the \gls{hmm} model based on the observed data and the forward and backward variables computed in the forward-backward procedure.
Given the forward and backward variables $\alpha_s(t)$ and $\beta_s(t)$, the update algorithm aims to maximize the likelihood of the observed data by adjusting the parameter values.

The update step is based on the expected sufficient statistics of the latent variables, which are the unobserved state sequences corresponding to the observations.

\subsubsection{Intermediate Variables}
We need to calculate the intermediate variables $\gamma_s(t)$ and $\xi_{ss'}(t)$, $\gamma_s(t)$ represent the expected number of times the model is in state $s$ at time $t$ and $\xi_{ss'}(t)$ represent the expected number of transitions from state $s$ to state $s'$ at time $t$.
For a given \gls{hmm} $\mathcal{M}$, the intermediate variables, $\gamma_s(t)$ and $\xi_{ss'}(t)$, are computed for each observation sequence $o_0, o_1, \dots, o_{|\mathbf{o}|-1} = \mathbf{o} \in \mathcal{O}$.
These variables are computed as follows:

\begin{equation}
    \gamma_s(t) = \frac{\alpha_s(t) \beta_s(t)}{\sum_{s' \in S} \;\alpha_{s'}(t) \beta_{s'}(t)}
    \label{eq:gamma}
\end{equation}

In \autoref{eq:gamma}, the numerator is the product of the forward variable $\alpha_s(t)$ and the backward variable $\beta_s(t)$, representing the joint likelihood of observing the enire sequence and the model given that the model passed by state $s$ at time $t$.
The denominator represents the likelihood of the observation sequence.

\begin{equation}
    \xi_{ss'}(t) = \frac{\alpha_s(t) P_{ss'} \omega_{s'}(t + 1) \beta_{s'}(t + 1)}{\sum_{s''}\alpha_{s''}(t) \beta_{s''}(t)}
    %{\sum_{s'' \in S} \;(\sum_{s''' \in S} \; (\alpha_{s''}(t) \; P_{s''s'''} \; \omega_{s'''}(t + 1) \; \beta_{s'''}(t + 1)))}
    \label{eq:xi}
\end{equation}


In \autoref{eq:xi}, the numerator is the joint likelihood of observing the sequence given that the model transitions from state $s$ to state $s'$ at time $t$.
The denominator represents the likelihood of the observation sequence.


The terms $\gamma_s(t)$ and $\xi_{ss'}(t)$ are normalized to ensure they represent probabilities.
For $\gamma_s(t)$, this involves dividing by the total likelihood across all states at time $t$, while for $\xi_{ss'}(t)$, normalization occurs over all possible transitions at time $t$.

\subsubsection{Parameter Update}
The parameter update step refines the parameter values of the model based on the earlier computed intermediate variables $\gamma_s(t)$ and $\xi_{ss'}(t)$.
The update algorithm aims to maximize the expected likelihood of the observed data under the model by adjusting the parameter values.

Once $\gamma_s(t)$ and $\xi_{ss'}(t)$ are computed for all states $s, s'$ and all time steps $t$ for every observation sequence, the model parameters can be updated to maximize the expected log-likelihood.

\paragraph*{\textit{Transition Probabilities ($P$)}}

We update the transition probabilities based on the expected number of transitions between states:


\begin{equation}
    P_{s \rightarrow s'} = \frac{\sum_{t = 1}^{|\mathbf{o}|-1} \xi_{ss'}(t)}{\sum_{t = 1}^{|\mathbf{o}|-1} \gamma_s(t)}
    \label{eq:transition-probabilities}
\end{equation}


The numerator sums the expected number of transitions from state $s$ to state $s'$ over all time steps.
The denominator sums the expected number of times the model is in state $s$ over all time steps, ensuring $P_{ss'}$ is normalized across all $s'$.

\paragraph*{\textit{Emission Probabilities ($\omega$)}}

We update the emission probabilities based on the expected occupancy of state $s$ and the corresponding observations, meaning the likelihood of observing the specific label $o$ in state $s$. 
The update is given by:
\begin{equation}
    \omega_s(o) = \frac{\sum_{t = 1}^{|\mathbf{o}|-1} \gamma_s(t) \lBrack o_t = o \rBrack}{\sum_{t = 1}^{|\mathbf{o}|-1} \gamma_s(t)}
    \label{eq:omega}
\end{equation}

The numerator sums $\gamma_s(t)$ for all time steps $t$ where the observed value $o_t = o$., meaning the model is in state $s$ and emits the observation $o$.
The denominator sums $\gamma_s(t)$ for all time steps $t$ where the model is in a given state $s$.


\paragraph*{\textit{Initial Probabilities ($\pi$)}}

We update the initial probabilities based on the expected occupancy of state $s$ at $t = 1$:
\begin{equation}
    \pi_s = \gamma_s(1)
    \label{eq:initial-probabilities}
\end{equation}

We can then update the parameters $\mathbf{x}$ by maximizing the expected log-likelihood of the observed data under the model.
The update algorithm iteratively refines the parameter values until convergence is reached.

\subsection{Matrix Operations}\label{subsec:matrix-operations}
%Baum-Welch with matrix operations
The Baum-Welch algorithm can be implemented using matrix operations to efficiently compute the forward and backward variables, intermediate variables, and parameter updates.

Given a \gls{hmm} $\mathcal{M}$ with parameters $\omega$, $P$, and $\pi$, and an observation sequence $\mathbf{o}$, the forward and backward variables $\alpha_t$ and $\beta_t$ can be computed using matrix operations as follows:

\begin{equation}
    \label{eq:alpha}
    \alpha_t =
    \begin{cases}
        \omega_0 \; \circ \; \pi   & \text{if } t = 0    \\
        \omega_t \; \circ \; \left( {P}^\top \alpha_{t - 1} \right)   & \text{if } 0 < t \leq |\mathbf{o}|-1 \\
    \end{cases}
\end{equation}


\begin{equation}
    \label{eq:beta}
    {\beta}_t =
    \begin{cases}
        \mathbbm{1} & \text{if } t = |\mathbf{o}|-1        \\
        {P} \; ({\beta}_{t + 1} \; \circ \; {\omega}_{t + 1}) & \text{if } 0 \leq t < |\mathbf{o}|-1 \\
    \end{cases}
\end{equation}

Here $\circ$ represents the Hadamard (point-wise) matrix multiplication, ${P}^\top$ denotes the transpose of the matrix ${P}$, and $\mathbbm{1}$ is a column vector of ones.
The resulting vectors ${\alpha}_t$ and ${\beta}_t$ for each time step $t$ are then related to $\alpha_s(t)$ and $\beta_s(t)$ for some $s$ by:

\begin{align}
    {\alpha}_t = \begin{bmatrix}
                                \alpha_{s_0}(t)       \\
                                \vdots                \\
                                \alpha_{s_{|S|-1}}(t) \\
    \end{bmatrix}, \;
    {\beta}_t = \begin{bmatrix}
                               \beta_{s_0}(t)       \\
                               \vdots               \\
                               \beta_{s_{|S|-1}}(t) \\
    \end{bmatrix}
\end{align}

$\gamma$ and $\xi$ can be expressed in terms of matrix operations as follows:

\begin{equation}
    {\gamma}_t = (\sum_{i=1}^{|\mathbf{o}|-1} (\alpha_{t i} \;\beta_{t i}))^{-1} \cdot \alpha_t \; \circ \; \beta_t
    \label{eq:gamma-matrix}
\end{equation}

\begin{equation}
    {\xi}_t = ((\sum_{i=1}^{|\mathbf{o}|-1} (\alpha_{t i} \; \beta_{t i}))^{-1} \cdot \; {P}) \; \circ \;(\alpha_t \otimes (\beta_{t+1} \; \circ \; {\omega}_{t+1}))
    \label{eq:xi-matrix}
\end{equation}

Here $\otimes$ represents the Kronecker (block) matrix multiplication, $\cdot$ denotes the scalar product and $^{-1}$ denotes the elementwise inverse of a matrix.

We can simplify $\sum_{i=1}^{|\mathbf{o}|-1} (\alpha_{t i} \beta_{t i})$ as, the sum does not depend on $t$:

\begin{align}
    \sum_{i=1}^{|\mathbf{o}|-1} (\alpha_{t i} \; \beta_{t i}) &= \sum_{i=1}^{|\mathbf{o}|-1} \alpha_{|\mathbf{o}|-1 i} \\
    &= \mathbbm{1}^T \; \alpha_{|\mathbf{o}|-1}
\end{align}

Here $\mathbbm{1}^T$ is a row vector of ones, and $\alpha_{|\mathbf{o}|-1}$ is the last column of the matrix ${\alpha_t}_{T \;\in 0\dots|\mathbf{o}|-1}$.

So we get:

\begin{equation}
    {\gamma}_t = (\mathbbm{1}^T \; \alpha_{|\mathbf{o}|-1})^{-1} \cdot \alpha_t \; \circ \; \beta_t
    \label{eq:gamma-matrix-ones}
\end{equation}

\begin{equation}
    {\xi}_t = ((\mathbbm{1}^T \; \alpha_{|\mathbf{o}|-1})^{-1} \cdot \; {P}) \; \circ \;(\alpha_t \otimes (\beta_{t+1} \; \circ \; {\omega}_{t+1}))
    \label{eq:xi-matrix-ones}
\end{equation}

The resulting vectors ${\gamma}_t$ and ${\xi}_t$ for each time step $t$ are then related to $\gamma_s(t)$ and $\xi_{ss'}(t)$ for some $s, s'$ by:

\begin{align}
    {\gamma}_t = \begin{bmatrix}
                                \gamma_{s_0}(t)       \\
                                \vdots                \\
                                \gamma_{s_{|S|-1}}(t) \\
    \end{bmatrix}, \;
    {\xi}_t = \begin{bmatrix}
                             \xi_{s_0 s_0}(t)      & \cdots & \xi_{s_0 s_{|S|-1}}(t)      \\
                             \vdots                & \ddots & \vdots                      \\
                             \xi_{s_{|S|-1}s_0}(t) & \cdots & \xi_{s_{|S|-1}s_{|S|-1}}(t) \\
    \end{bmatrix}
\end{align}

We can update the parameters with matrix operations as follows:

\begin{equation}
    {P} = (\mathbbm{1} \oslash \gamma) \bullet \xi
    \label{eq:transition-probabilities-update}
\end{equation}

\begin{equation}
    {\omega}_s(o) = (\mathbbm{1} \oslash \gamma) \bullet (\sum_{t=1}^{|\mathbf{o}|-1} \gamma_t \otimes \mathbbm{1}_{yt}^{|\mathbf{o}|-1})
    \label{eq:omega-update}
\end{equation}

\begin{equation}
    {\pi} = {\gamma}_1
    \label{eq:initial-probabilities-update}
\end{equation}

Where $\oslash$ denotes Hadamard division (elementwise division) product and $\bullet$ denotes the Katri-Rao product (column-wise Kronecker product).
In the formulas above, $\mathbbm{1}$ denotes a column vector of ones, $\mathbbm{1}_{yt}$ denotes a row vector of ones, $\gamma$ and $\xi$ are the sum of the respective vectors over all time steps $t$:
\begin{align}
    \gamma = \sum_{t=1}^{|\mathbf{o}|-1} \gamma_t, \;
    \xi = \sum_{t=1}^{|\mathbf{o}|-1} \xi_t
\end{align} 

