\section{Preliminaries}\label{sec:preliminaries}
In this section, we introduce the necessary background concepts and definitions that are essential for understanding the subsequent sections.
We begin by defining the key concepts of a \gls{hmm} and then describe how a \gls{hmm} can be represented using matrices.
We then introduce the Baum-Welch algorithm, which is used to estimate the parameters of a \gls{hmm} from observed data.
We describe all the steps involved in the Baum-Welch algorithm, including the forward-backward algorithm and the update algorithm.
Finally, we discuss how the Baum-Welch algorithm can be implemented using matrix operations to efficiently compute the forward and backward variables, intermediate variables, and parameter updates.

\subsection{Hidden Markov Models}\label{subsec:hmm}
%definition of HMM
\glspl{hmm} were introduced by Baum and Petrie in 1966~\cite{baum1966statistical}.
\gls{hmm} are a class of probabilistic graphical models that are widely used to model sequences of observations with underlying hidden states.
These models consist of two main components: observations and hidden states.
The observations are the visible data emitted by the model, while the hidden states represent the underlying process that generates these observations.
The objective of an \gls{hmm} is to infer the hidden states based on the observations.
\glspl{hmm} have applications in fields such as speech recognition~\cite{chavan2013overview}, bioinformatics~\cite{de2007hidden}, and natural language processing~\cite{murveit1990integrating}.
\glspl{hmm} was chosen as the model of choice for this project due to its versatility and ability to model complex systems.

\begin{definition}[Hidden Markov Model]
    A Hidden Markov Model (HMM) is a tuple $\mathcal{M} = (S, \mathcal{L}, \mathscr{l}, \tau,  \pi)$, where:
    \begin{itemize}
        \item $S$ is a finite set of states.
        \item $\mathcal{L}$ is a finite set of labels.
        \item $\mathscr{l}: S \rightarrow D(\mathcal{L})$ is the emission function.
        \item $\tau: S \rightarrow D(S)$ is the transition function.
        \item $\pi: S \rightarrow \mathbb{R}$ is the initial distribution.
    \end{itemize}
\end{definition}

Here, $D()$ denotes the set of probability distributions over a finite set. The model emits a label $l$ in state $s$ with probability $\mathscr{l}(s, l)$, transitions between states with probability $\tau(s, s')$, and starts in state $s$ with probability $\pi(s)$.

%should i write about it is descrite time or not?
\glspl{hmm} can be further classified into discrete-time and continuous-time models based on the time scale of the underlying process.
Discrete-time \glspl{hmm} are the most common type of \glspl{hmm} and are used to model sequences of observations that occur at discrete time intervals.
Continuous-time \glspl{hmm} are used to model sequences of observations that occur continuously over time.
In this project, we focus on discrete-time \glspl{hmm}.

%HMM as matrices
\subsection{Matrix Representation of HMMs}\label{subsec:matrix-representation}
\glspl{hmm} can be represented using matrices.
The emission function $\mathscr{l}$ can be represented as a matrix $\omega$ where $\omega_{s, l} = \mathscr{l}(s, l)$.
The matrix $\omega$ has the size $|S| \times |\mathcal{L}|$.
The sum of each row in the matrix $\omega$ is equal to one, reflecting the total probability of emitting all labels from a given state.
\[
\omega = \begin{bmatrix}
    \mathscr{l}(s_1, l_1) & \cdots & \mathscr{l}(s_1, l_{|\mathcal{L}|}) \\
    \vdots & \ddots & \vdots \\
    \mathscr{l}(s_{|S|}, l_1)  & \cdots & \mathscr{l}(s_{|S|}, l_{|\mathcal{L}|})
\end{bmatrix}
\]
The transition function $\tau$ can be represented as a matrix $P$ where $P_{s, s'} = \tau(s, s')$.
The matrix $P$ has the size $|S| \times |S|$.
The sum of each row in $P$ is equal to one, reflecting the total probability of transitioning from a given state to all other states.
\[
P = \begin{bmatrix}
    \tau(s_1, s_1) &  \cdots & \tau(s_1, s_{|S|}) \\
    \vdots  & \ddots & \vdots \\
    \tau(s_{|S|}, s_1) & \cdots & \tau(s_{|S|}, s_{|S|})
\end{bmatrix}
\]
The initial distribution $\pi$ can be represented as a vector $\pi$ where $\pi_s = \pi(s)$.
The vector $\pi$ has the size $|S|$.
The sum of all elements in $\pi$ is equal to one, reflecting the total probability of starting in each state.
\[
\pi = \begin{bmatrix}
    \pi(s_1) \\
    \vdots \\
    \pi(s_{|S|})
\end{bmatrix} 
\]

\subsection{Observations and Hidden States}\label{subsec:observations-hidden-states}
An \gls{hmm} operates on a multiset of observations, denoted as  $\mathcal{O} = {O_1, O_2, \ldots, O_N}$, where each $O_i$ is a observation sequence of labels $\mathbf{O} = o_1, o_2, \ldots, o_{|\mathbf{O}|-1}$.
The task is to infer the sequence of hidden states $S = s_1, s_2, \ldots, s_{|\mathbf{O}|-1}$ that most likely generated these observations.

This involves maximizing the probability of the hidden states $S$ conditioned on the observed sequence $O$. In essence, the observations are the labels for which we seek the corresponding hidden state sequence, representing the most probable path that generated these observations.

This inference is commonly performed using the Baum-Welch algorithm, a widely used method for estimating the probabilities of an \gls{hmm} and finding the most likely hidden state sequence.

%Baum-Welch for HMM mathamatically
\subsection{Baum-Welch Algorithm}\label{subsec:baum-welch}
The Baum-Welch algorithm is a fundamental method for estimating the parameters of a \gls{hmm} given a sequence of observations. 
These parameters include the emission matrix $\omega$, the transition matrix $P$, and the initial state distribution $\pi$.
The algorithm is widely recognized as the standard approach for training \glspl{hmm} and was chosen for this project due to its ability to estimate these parameters without prior knowledge of the hidden states that generated the observations.

The Baum-Welch algorithm applies the Expectation-Maximization (EM) framework to iteratively improve the likelihood of the observed data under the current model parameters. It consists of the following steps:

\begin{enumerate}
    \item \textbf{Initialization:} Begin with initial estimates for the \gls{hmm} parameters $(\pi, P, \omega)$.
    \item \textbf{Expectation Step (E-step):}
    Compute the forward probabilities $\alpha_s(t)$ and backward probabilities $\beta_s(t)$, which represent:
        \begin{itemize}
            \item The probability of observing the sequence up to time $t$, given that the \gls{hmm} is in state $s$ at time $t$ ($\alpha_s(t)$).
            \item The probability of observing the sequence from time $t+1$ to the end, given that the \gls{hmm} is in state $s$ at time $t$ ($\beta_s(t)$).
        \end{itemize}

    \item \textbf{Maximization Step (M-step):} Update the \gls{hmm} parameters $(\pi, P, \omega)$ to maximize the likelihood of the observed data based on the expected counts computed in the E-step.
    \item \textbf{Iteration:} Repeat the E-step and M-step until convergence, i.e., when the change in likelihood between iterations falls below a predefined threshold.
    
\end{enumerate}
The Baum-Welch algorithm refines the \gls{hmm} parameters by iteratively maximizing the likelihood of the observations. 
Starting with an initial hypothesis $\textbf{x}_0 = (\pi, P, \omega)$, the algorithm produces a sequence of parameter estimates $\textbf{x}_1, \textbf{x}_2, \ldots$, where each new estimate improves upon the previous one. 
The process terminates when the improvement in likelihood is sufficiently small, satisfying the convergence criterion:
\[
||l(o,x_n)|| < \epsilon
\]

Where $l(o, x_n)$ is the likelihood of the observations under the parameters $x_n$, and $\epsilon > 0$ is a small threshold.
%where $l(\textbf{x}_n)$ is the likelihood of the observations under the parameters $\textbf{x}_n$, and $\epsilon > 0$ is a small threshold.

The steps of the Baum-Welch algorithm are detailed in the following sections, including the initialization of the \gls{hmm} parameters, the forward-backward algorithm, and the update algorithm, see \autoref{subsec:initialization}, \ref{subsec:forward-backwards_algorithm}, and \ref{subsec:update-algorithm} respectively.

\subsection{Initialization of HMM Parameters}\label{subsec:initialization}
Before starting the Baum-Welch algorithm, we need to initialize the model parameters: The emission matrix $\omega$, the transition matrix $P$, and the initial state distribution $\pi$.

Since the algorithm is designed to converge iteratively toward an optimal parameter set, the initial estimates do not need to be exact. 
However, reasonable initialization can accelerate convergence and improve numerical stability~\cite{benyacoub2015initial}. 
As we are working with \glspl{hmm} we do not know which labels are emitted by which states, and we do not know the transition probabilities between states,therefore we cannot initialize the parameters based on some prior knowledge.
A common approach to initialize these parameters is as follows:

\textit{Initialization state probability distribution}:

The initial state probabilities $\pi$ represent the likelihood of starting in each hidden state $s \in S$. 
To estimate initial state probabilities, we can use one of the following strategies:
\begin{itemize}
    % \item Examine the observed data: Estimate the initial probabilities based on the observed data. For example, if certain states are more likely to occur at the beginning of the sequence, assign higher probabilities to these states. 
    % $\pi_s = \frac{R_s}{\sum_{s' \in S} R_{s'}}$ where $R_s$ is the number of times state $s$ occurs at the beginning of the sequence.
    \item Random initialization: Assign random probabilities to each state $s \in S$, such that $\sum_{s \in S} \pi_s = 1$.
    \item Uniform initialization: Set $\pi_s = \frac{1}{|S|}$ for all $s \in S$.
\end{itemize}

\textit{Initialization transition probability distribution}:

The transition matrix $P$ represents the probability of transitioning from one state to another. 
To estimate transition probabilities, we can use one of the following strategies:
\begin{itemize}
    % \item Examine the observed data: Estimate the transition probabilities based on the observed data. For example, if certain states are more likely to transition to others, assign higher probabilities to these transitions. 
    % $P_{s \rightarrow s'} = \frac{C_{s \rightarrow s'}}{\sum_{s'' \in S} C_{s \rightarrow s''}}$ where $C_{s \rightarrow s'}$ is the number of transitions from state $s$ to state $s'$.
    \item Random initialization: Assign random probabilities to each transition $P_{s \rightarrow s'}$, such that $\sum_{s' \in S} P_{s \rightarrow s'} = 1$.
    \item Uniform initialization: Set $P_{s \rightarrow s'} = \frac{1}{|S|}$ for all $s, s' \in S$.
\end{itemize}

\textit{Initialization emission probability distribution}:

The emission matrix $\omega$ represents the probability of emitting each observation label from each hidden state. 
To estimate emission probabilities, we can use one of the following strategies:
\begin{itemize}
    % \item Examine the observed data: Estimate the emission probabilities based on the observed data. For example, if certain states are more likely to emit certain labels, assign higher probabilities to these emissions. 
    % $\omega_{s, l} = \frac{C_{s, l}}{\sum_{l' \in \mathcal{L}} C_{s, l'}}$ where $C_{s, l}$ is the number of times state $s$ emits label $l$.
    \item Random initialization: Assign random probabilities to each emission $\omega_{s, l}$, such that $\sum_{l \in \mathcal{L}} \omega_{s, l} = 1$ for all $s \in S$.
    \item Uniform initialization: Set $\omega_{s, l} = \frac{1}{|\mathcal{L}|}$ for all $s \in S, l \in \mathcal{L}$.
\end{itemize}

We initialize the parameters using uniform initialization for the emission matrix $\omega$, the transition matrix $P$, and the initial state distribution $\pi$.

These initialization strategies provide a starting point for the Baum-Welch algorithm to iteratively refine the model parameters based on the observed data.

\subsection{Forward-Backward Algorithm}\label{subsec:forward-backwards_algorithm}
%Forward-Backward algorithm
For a given \gls{hmm} $\mathcal{M}$, the forward-backward algorithm computes the forward and backward variables, $\alpha_s(t)$ and $\beta_s(t)$, for each observation sequence $o_0, o_1, \dots, o_{|\mathbf{o}|-1} = \mathbf{o} \in \mathcal{O}$.
The forward variable $\alpha_s(t)$ represents the likelihood of observing the partial sequence $o_0, o_1, \dots, o_t$ and being in state $s$ at time $t$, given the model $\mathcal{M}$.
The backward variable $\beta_s(t)$ represents the likelihood of observing the partial sequence $o_{t+1}, o_{t+2}, \dots, o_{|\mathbf{o}|-1}$ given state $s$ at time $t$ and the model $\mathcal{M}$.


% \begin{equation}
%     \alpha_s(t) = l(o_0, o_1, \dots, o_t, S_{t} = s \mid \mathcal{M})
%     \label{eq:alpha-recursive}
% \end{equation}


% \begin{equation}
%     \beta_s(t) = l(o_{t+1}, o_{t+2}, \dots, o_{|\mathbf{o}|-1} \mid S_{t} = s, \mathcal{M})
%     \label{eq:beta-recursive}
% \end{equation}

The forward variable $\alpha_s(t)$ and backward variable $\beta_s(t)$ can be computed recursively as follows:

\begin{equation}
    \alpha_s(t) =
    \begin{cases}
        \omega_s(0) \; \pi_s & \text{if } t = 0 \\
        \omega_s(t) \sum_{s' \in S} P_{s's}\alpha_{s'}(t - 1) & \text{if } 0 < t \leq |\mathbf{o}| - 1 \\
    \end{cases}
    \label{eq:forward-recursive}
\end{equation}


\begin{equation}
    \beta_s(t) =
    \begin{cases}
        \mathbbm{1} & \text{if } t = |\mathbf{o}| - 1 \\
        \sum_{s' \in S} P_{ss'} \omega_{s'}(t + 1) \beta_{s'}(t + 1) & \text{if } 0 \leq t < |\mathbf{o}| - 1 \\
    \end{cases}
    \label{eq:backward-recursive}
\end{equation}


Here, $\omega_{s}(t)$ is the likelihood of observing $o_t$ given that the state at time $t$ is $s$ and the model $\mathcal{M}$, formally $\omega_s(t) = l(o_t \mid S_t = s, \mathcal{M})$.
Meaning that $\omega_s(t)$ is the probability of observing the label $o_t$ in state $s$.

The forward-backward algorithm computes the forward and backward variables for each state $s$ and time $t$ in the observation sequence $\mathbf{o}$, providing a comprehensive view of the likelihood of the observed data under the model.

In preparation for later discussions we would like to draw the attention to the fact that the above recurrences can be solved using dynamic programming requiring one to use $\Theta(|S|\times|(|\mathbf{o}|-1)|)$ space.


\subsection{Update Algorithm}\label{subsec:update-algorithm}
%Update of HMM
The update algorithm refines the parameter values of the \gls{hmm} model based on the observed data and the forward and backward variables computed in the forward-backward procedure.
Given the forward and backward variables $\alpha_s(t)$ and $\beta_s(t)$, the update algorithm aims to maximize the likelihood of the observed data by adjusting the parameter values.

The update step is based on the expected sufficient statistics of the latent variables, which are the unobserved state sequences corresponding to the observations.

\subsubsection{Intermediate Variables}
We need to calculate the intermediate variables $\gamma_s(t)$ and $\xi_{ss'}(t)$, $\gamma_s(t)$ represent the expected number of times the model is in state $s$ at time $t$ and $\xi_{ss'}(t)$ represent the expected number of transitions from state $s$ to state $s'$ at time $t$.
For a given \gls{hmm} $\mathcal{M}$, the intermediate variables, $\gamma_s(t)$ and $\xi_{ss'}(t)$, are computed for each observation sequence $o_0, o_1, \dots, o_{|\mathbf{o}|-1} = \mathbf{o} \in \mathcal{O}$.
These variables are computed as follows:

\begin{equation}
    \gamma_s(t) = \frac{\alpha_s(t) \beta_s(t)}{\sum_{s' \in S} \;\alpha_{s'}(t) \beta_{s'}(t)}
    \label{eq:gamma}
\end{equation}

In \autoref{eq:gamma}, the numerator is the product of the forward variable $\alpha_s(t)$ and the backward variable $\beta_s(t)$, representing the joint likelihood of observing the enire sequence and the model given that the model passed by state $s$ at time $t$.
The denominator represents the likelihood of the observation sequence.

\begin{equation}
    \xi_{ss'}(t) = \frac{\alpha_s(t) P_{ss'} \omega_{s'}(t + 1) \beta_{s'}(t + 1)}
    {\sum_{s''}\alpha_{s''}(t) \beta_{s''}(t)}
    %{\sum_{s'' \in S} \;(\sum_{s''' \in S} \; (\alpha_{s''}(t) \; P_{s''s'''} \; \omega_{s'''}(t + 1) \; \beta_{s'''}(t + 1)))}
    \label{eq:xi}
\end{equation}


In \autoref{eq:xi}, the numerator is the joint likelihood of observing the sequence given that the model transitions from state $s$ to state $s'$ at time $t$.
The denominator represents the likelihood of the observation sequence.


The terms $\gamma_s(t)$ and $\xi_{ss'}(t)$ are normalized to ensure they represent probabilities.
For $\gamma_s(t)$, this involves dividing by the total likelihood across all states at time $t$, while for $\xi_{ss'}(t)$, normalization occurs over all possible transitions at time $t$.

\subsubsection{Parameter Update}
The parameter update step refines the parameter values of the model based on the earlier computed intermediate variables $\gamma_s(t)$ and $\xi_{ss'}(t)$.
The update algorithm aims to maximize the expected likelihood of the observed data under the model by adjusting the parameter values.

Once $\gamma_s(t)$ and $\xi_{ss'}(t)$ are computed for all states $s, s'$ and all time steps $t$ for every observation sequence, the model parameters can be updated to maximize the expected log-likelihood.

\paragraph*{\textit{Transition Probabilities ($P$)}}

We update the transition probabilities based on the expected number of transitions between states:


\begin{equation}
    P_{s \rightarrow s'} = \frac{\sum_{t = 1}^{|\mathbf{o}|-1} \xi_{ss'}(t)}{\sum_{t = 1}^{|\mathbf{o}|-1} \gamma_s(t)}
    \label{eq:transition-probabilities}
\end{equation}


The numerator sums the expected number of transitions from state $s$ to state $s'$ over all time steps.
The denominator sums the expected number of times the model is in state $s$ over all time steps, ensuring $P_{ss'}$ is normalized across all $s'$.

\paragraph*{\textit{Emission Probabilities ($\omega$)}}

We update the emission probabilities based on the expected occupancy of state $s$ and the corresponding observations, meaning the likelihood of observing the specific label $o$ in state $s$. 
The update is given by:
\begin{equation}
    \omega_s(o) = \frac{\sum_{t = 1}^{|\mathbf{o}|-1} \gamma_s(t) \lBrack o_t = o \rBrack}{\sum_{t = 1}^{|\mathbf{o}|-1} \gamma_s(t)}
    \label{eq:omega}
\end{equation}

The numerator sums $\gamma_s(t)$ for all time steps $t$ where the observed value $o_t = o$., meaning the model is in state $s$ and emits the observation $o$.
The denominator sums $\gamma_s(t)$ for all time steps $t$ where the model is in a given state $s$.


\paragraph*{\textit{Initial Probabilities ($\pi$)}}

We update the initial probabilities based on the expected occupancy of state $s$ at $t = 1$:
\begin{equation}
    \pi_s = \gamma_s(1)
    \label{eq:initial-probabilities}
\end{equation}

We can then update the parameters $\mathbf{x}$ by maximizing the expected log-likelihood of the observed data under the model.
The update algorithm iteratively refines the parameter values until convergence is reached.

\subsection{Matrix Operations}\label{subsec:matrix-operations}
%Baum-Welch with matrix operations
The Baum-Welch algorithm can be implemented using matrix operations to efficiently compute the forward and backward variables, intermediate variables, and parameter updates.

Given a \gls{hmm} $\mathcal{M}$ with parameters $\omega$, $P$, and $\pi$, and an observation sequence $\mathbf{o}$, the forward and backward variables $\alpha_t$ and $\beta_t$ can be computed using matrix operations as follows:

\begin{equation}
    \label{eq:alpha}
    \alpha_t =
    \begin{cases}
        \omega_0 \; \circ \; \pi   & \text{if } t = 0    \\
        \omega_t \; \circ \; \left( {P}^\top \alpha_{t - 1} \right)   & \text{if } 0 < t \leq |\mathbf{o}|-1 \\
    \end{cases}
\end{equation}


\begin{equation}
    \label{eq:beta}
    {\beta}_t =
    \begin{cases}
        \mathbbm{1} & \text{if } t = |\mathbf{o}|-1        \\
        {P} \; ({\beta}_{t + 1} \; \circ \; {\omega}_{t + 1}) & \text{if } 0 \leq t < |\mathbf{o}|-1 \\
    \end{cases}
\end{equation}

Here $\circ$ represents the Hadamard (point-wise) matrix multiplication, ${P}^\top$ denotes the transpose of the matrix ${P}$, and $\mathbbm{1}$ is a column vector of ones.
The resulting vectors ${\alpha}_t$ and ${\beta}_t$ for each time step $t$ are then related to $\alpha_s(t)$ and $\beta_s(t)$ for some $s$ by:

\begin{align}
    {\alpha}_t = \begin{bmatrix}
        \alpha_{s_0}(t)       \\
        \vdots                \\
        \alpha_{s_{|S|-1}}(t) \\
    \end{bmatrix}, \;
    {\beta}_t = \begin{bmatrix}
        \beta_{s_0}(t)       \\
        \vdots               \\
        \beta_{s_{|S|-1}}(t) \\
    \end{bmatrix}
\end{align}

$\gamma$ and $\xi$ can be expressed in terms of matrix operations as follows:

\begin{equation}
    {\gamma}_t = (\sum_{i=1}^{|\mathbf{o}|-1} (\alpha_{t i} \;\beta_{t i}))^{-1} \cdot \alpha_t \; \circ \; \beta_t
    \label{eq:gamma-matrix}
\end{equation}

\begin{equation}
    {\xi}_t = ((\sum_{i=1}^{|\mathbf{o}|-1} (\alpha_{t i} \; \beta_{t i}))^{-1} \cdot \; {P}) \; \circ \;(\alpha_t \otimes (\beta_{t+1} \; \circ \; {\omega}_{t+1}))
    \label{eq:xi-matrix}
\end{equation}

Here $\otimes$ represents the Kronecker (block) matrix multiplication, $\cdot$ denotes the scalar product and $^{-1}$ denotes the elementwise inverse of a matrix.

We can simplify $\sum_{i=1}^{|\mathbf{o}|-1} (\alpha_{t i} \beta_{t i})$ as, the sum does not depend on $t$:

\begin{align}
    \sum_{i=1}^{|\mathbf{o}|-1} (\alpha_{t i} \; \beta_{t i}) &= \sum_{i=1}^{|\mathbf{o}|-1} \alpha_{|\mathbf{o}|-1 i} \\
    &= \mathbbm{1}^T \; \alpha_{|\mathbf{o}|-1}
\end{align}

Here $\mathbbm{1}^T$ is a row vector of ones, and $\alpha_{|\mathbf{o}|-1}$ is the last column of the matrix ${\alpha_t}_{T \;\in 0\dots|\mathbf{o}|-1}$.

So we get:

\begin{equation}
    {\gamma}_t = (\mathbbm{1}^T \; \alpha_{|\mathbf{o}|-1})^{-1} \cdot \alpha_t \; \circ \; \beta_t
    \label{eq:gamma-matrix-ones}
\end{equation}

\begin{equation}
    {\xi}_t = ((\mathbbm{1}^T \; \alpha_{|\mathbf{o}|-1})^{-1} \cdot \; {P}) \; \circ \;(\alpha_t \otimes (\beta_{t+1} \; \circ \; {\omega}_{t+1}))
    \label{eq:xi-matrix-ones}
\end{equation}

The resulting vectors ${\gamma}_t$ and ${\xi}_t$ for each time step $t$ are then related to $\gamma_s(t)$ and $\xi_{ss'}(t)$ for some $s, s'$ by:

\begin{align}
    {\gamma}_t = \begin{bmatrix}
        \gamma_{s_0}(t)       \\
        \vdots                \\
        \gamma_{s_{|S|-1}}(t) \\
    \end{bmatrix}, \;
    {\xi}_t = \begin{bmatrix}
        \xi_{s_0 s_0}(t)      & \cdots & \xi_{s_0 s_{|S|-1}}(t)      \\
        \vdots                & \ddots & \vdots                      \\
        \xi_{s_{|S|-1}s_0}(t) & \cdots & \xi_{s_{|S|-1}s_{|S|-1}}(t) \\
    \end{bmatrix}
\end{align}

We can update the parameters with matrix operations as follows:

\begin{equation}
    {P} = (\mathbbm{1} \oslash \gamma) \bullet \xi
    \label{eq:transition-probabilities-update}
\end{equation}

\begin{equation}
    {\omega}_s(o) = (\mathbbm{1} \oslash \gamma) \bullet (\sum_{t=1}^{|\mathbf{o}|-1} \gamma_t \otimes \mathbbm{1}_{yt}^{|\mathbf{o}|-1})
    \label{eq:omega-update}
\end{equation}

\begin{equation}
    {\pi} = {\gamma}_1
    \label{eq:initial-probabilities-update}
\end{equation}

Where $\oslash$ denotes Hadamard division (elementwise division) product and $\bullet$ denotes the Katri-Rao product (column-wise Kronecker product).
In the formulas above, $\mathbbm{1}$ denotes a column vector of ones, $\mathbbm{1}_{yt}$ denotes a row vector of ones, $\gamma$ and $\xi$ are the sum of the respective vectors over all time steps $t$:
\begin{align}
    \gamma = \sum_{t=1}^{|\mathbf{o}|-1} \gamma_t, \;
    \xi = \sum_{t=1}^{|\mathbf{o}|-1} \xi_t
\end{align} 

